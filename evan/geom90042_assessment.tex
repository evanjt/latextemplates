%% Assessment template for 
%% GEOM90042: Spatial Information Programming
%% 
%% Authored by Evan Thomas evan@evanjt.com
%% Last updated 13/10/2020

%% -- 2020
%% Released	
%% Due		
%% Weighting: 

%% Assessment variables


\documentclass[10pt,a4paper]{article}
\usepackage[english]{babel}
\setlength{\parindent}{0pt}
\usepackage[utf8]{inputenc}
\usepackage{fancyhdr}
\usepackage{graphicx}
\usepackage{multirow}
\usepackage{booktabs}
\usepackage{hyperref}
\graphicspath{{./images/}}
\hypersetup{
    colorlinks=true,
    linkcolor=blue,
    filecolor=magenta,      
    urlcolor=cyan,
}
\usepackage[a4paper, total={7in,9in}]{geometry}
\newcommand{\AssignmentName}{Assessment [NUMBER]}
\pagestyle{fancy}
\fancyhf{}
\usepackage[procnames]{listings}
\lstset{language=Python, 
        basicstyle=\ttfamily\small, 
        showstringspaces=false,
        procnamekeys={def,class}}
\rhead{GEOM90042: Spatial Information Programming}
\lhead{\AssignmentName}
\rfoot{Page \thepage}
\begin{document}

{
\centering 
\huge{\AssignmentName: [DESCRIPTION]}\par
}
\section{Introduction}
This is an [INDIVIDUAL/GROUP] assessment. It is worth [WEIGHTING] of your final class mark. No late submissions will be accepted.

\section{Aim}
The aim of this assessment is to ...

\section{Assessment}


\section{Questions}

%% Task 1
\subsection{Task 1: [DESCRIPTION]}

\begin{enumerate}
\item{Write a ...}

\textbf{Note}: ... 

%% Task 1, q2 
\item{Write a ...}
\item{Write a ...}

%--Sub-enumeration (START)--
\begin{enumerate}
\item{Import ...}
\item{Process a ...}
\end{enumerate}
%--Sub-enumeration (END)--
\end{enumerate}

% Task 2
\subsection{Task 2: [DESCRIPTION]}
\begin{enumerate}
\item{Write a ...}

\item{Write a ...}
\end{enumerate}

% Task 3
\subsection{Task 3: [DESCRIPTION]}

You will ...
\begin{enumerate}
\item{Compute the ...}
\item{Compute the ...}
\item{Compute the ...}
\end{enumerate}

% Code block example 
\begin{lstlisting}

	def main():
	    # Setup constant
	    csv_filename = "trajectory_data.csv"
	    filename = os.path.join(os.getcwd(), csv_filename)
		
	    # Run tasks
	    reprojected_data = task_1(filename)
	    task_2()
	    task_3(reprojected_data)
		
	    # Finished
	    print("Program complete")
		
\end{lstlisting}

% Sample data section
\section{Sample data}
\label{sec:sample_output}

Below is an example of ...

\section{Marking scheme}


Your assessment will be marked out of ...
\begin{itemize}
\item{Your program ...}
\item{Your program computes ...}
\item{Your program is ...}
\end{itemize}

Note that only actual code counts as lines. Comments are not counted as lines, and you will lose marks if your comments are too sparse or missing. A very rough guideline: a program with 150 lines of codes has also the same amount of comments. 
The table below shows detailed marking criteria associated with marks:\\

\begin{center}
\begin{tabular}{  l | l | c  }
\toprule
\textbf{Criteria} & \textbf{Sub-criteria} & \textbf{Mark}\\
\midrule
\multirow{2}{*}{Correctness} & Task completion, output correctness, unit tests pass, runs from console & 8\\
  & Interoperability, special case and exception handling, use of main() & 2 \\
%\hline
\\
%.sbn & \multirow{2}{*}{The spatial index} & ewdfs\\
\multirow{2}{*}{Design} & Abstraction of problem, appropriate use of functions & 3\\
 & Readability, elegance, code structure, PEP-8 conformity & 3\\
%\hline
\\
\multirow{2}{*}{Documentation} & Understandability of function/variable names (self-documenting) & 2\\
 & Commenting and documentation & 2\\
\bottomrule
\end{tabular}
\end{center}


\section{Tips}
[INSERT ANY TIPS HERE]

\section{Submission}
\label{sec:submission}
Submit a single \textbf{.zip} file (where \textbf{studentno} is your student number) containing the following files for submission:\\

\end{document}