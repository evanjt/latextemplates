%%%%%%%%%%%%%%%%%%%%%%%%%%%%%%%%%%%%%%%%%%%%%%%%%%%%%%%%%%%%%%%%%%%%%%
%
\section{Introduction}
%
%%%%%%%%%%%%%%%%%%%%%%%%%%%%%%%%%%%%%%%%%%%%%%%%%%%%%%%%%%%%%%%%%%%%%%

This is a small overview how to write papers at the IVT in \LaTeX.

If you see bugs and errors in the layout please contact
joseph.molloy@ivt.baug.ethz.ch.

%%%%%%%%%%%%%%%%%%%%%%%%%%%%%%%%%%%%%%%%%%%%%%%%%%%%%%%%%%%%%%%%%%%%%%
\subsection{Structuring}
%%%%%%%%%%%%%%%%%%%%%%%%%%%%%%%%%%%%%%%%%%%%%%%%%%%%%%%%%%%%%%%%%%%%%%

To structure you paper you can define several headings and
subheadings:
``\textbackslash{}section'', ``\textbackslash{}subsection'',
``\textbackslash{}subsubsection'', ``\textbackslash{}paragraph'' and
``\textbackslash{}subparagraph''. The first three will be numbered and
will show
up in the content. The layout defines the style of the headings.

Examples of sections and subsections:

\subsubsection{A subsubsection}

Some text. Some text. Some text. Some text. Some text. Some text. Some
text. Some text. Some text. Some text. Some text. Some text. Some
text. Some text. Some text. Some text.

\paragraph{A paragraph}

Some text. Some text. Some text. Some text. Some text. Some text. Some
text. Some text. Some text. Some text. Some text. Some text. Some
text. Some text. Some text. Some text.

\subparagraph{A subparagraph}

Some text. Some text. Some text. Some text. Some text. Some text. Some
text. Some text. Some text. Some text. Some text. Some text. Some
text. Some text. Some text. Some text.

\subsection{A Really Really Really Really Really Really Really Really Really Really Really Really Long Section title}
%%%%%%%%%%%%%%%%%%%%%%%%%%%%%%%%%%%%%%%%%%%%%%%%%%%%%%%%%%%%%%%%%%%%%%
\subsubsection{Page Break}
%%%%%%%%%%%%%%%%%%%%%%%%%%%%%%%%%%%%%%%%%%%%%%%%%%%%%%%%%%%%%%%%%%%%%%

There are three ways of performing a manual page break.
\begin{itemize}
  \item ``\textbackslash{}clearpage'': omits the remaining space of the
current page and starts the new one. Inserting two or more times that
command does NOT produce follow up empty pages.
  \item ``\textbackslash{}cleardoublepage'': It does the same than the
command above, but for a report \LaTeX\ type that define different
layouts for the odd and even pages (i.e.\ dissertation layouts), it
sometimes produces a complete follow up empty page such that the next
sections will occur always on the even page (odd page, resp.)
  \item ``\textbackslash{}include'': This is another way to start on a
new page. If you use the ``\textbackslash{}include''
command for separating a paper
into different \texttt{.tex} files,
``\textbackslash{}include'' will always start on
top of the next page. If you do not want that, but still want to
separate the paper into different files,
use the ``\textbackslash{}input''
command instead.
\end{itemize}

\clearpage

\cleardoublepage

%%%%%%%%%%%%%%%%%%%%%%%%%%%%%%%%%%%%%%%%%%%%%%%%%%%%%%%%%%%%%%%%%%%%%%
\subsection{Text Blocks}
%%%%%%%%%%%%%%%%%%%%%%%%%%%%%%%%%%%%%%%%%%%%%%%%%%%%%%%%%%%%%%%%%%%%%%

A newline (one time ``Return'') does NOT produce a new Block.
You have to separate blocks with a COMPLETE EMPTY line.

Hence, it is a good idea to insert hard line breaks frequently,
e.g.~after commas or full stops.
Most version control systems such as SVN are line-based;
changes can be tracked easier if a paragraph is split
over several lines.
By the same token, it is not a good idea to let your text editor
manage the line breaks for you.

This is a block. This is a block. This is a block. This is a block.
This is a block. This is a block. This is a block. This is a block.
This is a block. This is a block. This is a block. This is a block.
This is a block. This is a block. This is a block.

This is another block. This is another block. This is another block.
This is another block. This is another block. This is another block.
This is another block. This is another block. This is another block.
This is another block. This is another block. This is another block.
This is another block. This is another block. This is another block.

%%%%%%%%%%%%%%%%%%%%%%%%%%%%%%%%%%%%%%%%%%%%%%%%%%%%%%%%%%%%%%%%%%%%%%
\subsection{Lists}
%%%%%%%%%%%%%%%%%%%%%%%%%%%%%%%%%%%%%%%%%%%%%%%%%%%%%%%%%%%%%%%%%%%%%%

There are three ways of making lists:

\subsubsection{Items}

\begin{itemize}
  \item Item one
  \begin{itemize}
    \item Item one
    \item Item two Item two Item two Item two Item two Item two Item
    two Item two
    Item two Item two Item two Item two Item two Item two Item
    two Item two
    \item Item three
  \end{itemize}
  \item Item two Item two Item two Item two Item two Item two Item two
Item twoItem two Item two Item two Item two Item two Item two Item two
Item two
  \item Item three
\end{itemize}

\subsubsection{Enumerations}

\begin{enumerate}
  \item Item one
  \begin{enumerate}
    \item Item one
    \item Item two Item two Item two Item two Item two Item two Item
two Item twoItem two Item two Item two Item two Item two Item two Item
two Item two
    \item Item three
  \end{enumerate}
  \item Item two Item two Item two Item two Item two Item two Item two
Item twoItem two Item two Item two Item two Item two Item two Item two
Item two
  \item Item three
\end{enumerate}

\subsubsection{Descriptive list}

\begin{description}
  \item[Desc one] Item one
  \begin{description}
    \item[Desc one] Item one
    \item[Desc two] Item two Item two Item two Item two Item two Item
two Item two Item two Item two Item two Item two Item two Item two
Item two Item two Item two Item two
    \item[Desc three] Item three
  \end{description}
  \item[Desc two] Item two Item two Item two Item two Item two Item
two Item two Item two Item two Item two Item two Item two Item two
Item two Item two Item two Item two
  \item[Desc three] Item three
\end{description}

\subsubsection{Sub-Lists}

List can be nested in any fashion you like:

\begin{itemize}
  \item Item one
  \begin{itemize}
    \item Item one
    \item Item two
    \item Item three
  \end{itemize}
  \item Item three
  \begin{enumerate}
    \item Item one
    \begin{description}
      \item[Desc one] Item one
      \item[Desc two] Item two Item two Item two Item two Item two
Item two Item two Item two Item two Item two Item two Item two Item
two Item two Item two Item two Item two
      \begin{itemize}
        \item Item one
        \item Item two Item two Item two Item two Item two Item
        two Item two Item two Item two Item two Item two Item two
        Item two Item
        two Item two Item two Item twoItem two
        \item Item three
      \end{itemize}
      \item[Desc three] Item three
      \begin{enumerate}
        \item Item one
        \item Item two Item two Item two Item two Item two Item two
        Item two Item two Item two Item two Item two Item two
        Item two Item
        two Item two Item two Item two Item two Item two Item two Item two Item
        two
        \item Item three
      \end{enumerate}
    \end{description}
    \item Item two
    \item Item three
  \end{enumerate}
  \item Item four
\end{itemize}

%%%%%%%%%%%%%%%%%%%%%%%%%%%%%%%%%%%%%%%%%%%%%%%%%%%%%%%%%%%%%%%%%%%%%%
\subsection{Text appearance}
%%%%%%%%%%%%%%%%%%%%%%%%%%%%%%%%%%%%%%%%%%%%%%%%%%%%%%%%%%%%%%%%%%%%%%

There are also more commands to change the appearance of text.
However, underlining and letter-spacing
is not recommended when writing papers.
Some even consider bold-face as bad practice, as it changes
the ``average greyness'' of the text
(as does underlining and letter-spacing).

\subsubsection{Bold, etc.}

\emph{Emphasis (instead of italic)}, \textbf{Bold},
\textsc{Small caps}, \textbf{\textit{Bold italics}},
\texttt{Teletype}, \textsf{Sans serif}.

\subsubsection{URL}

To write an url type: \url{www.ivt.ethz.ch}.
The PDF will contain a link to that URL.

%%%%%%%%%%%%%%%%%%%%%%%%%%%%%%%%%%%%%%%%%%%%%%%%%%%%%%%%%%%%%%%%%%%%%%
\subsection{Special Characters}
%%%%%%%%%%%%%%%%%%%%%%%%%%%%%%%%%%%%%%%%%%%%%%%%%%%%%%%%%%%%%%%%%%%%%%

\subsubsection{Large Space}

Two sentence are separated with a large empty space. For dots used in
acronyms (``i.e.'' or ``e.g.'') followed by an empty space, you do not
want this extra large space. In this case you should add a backslash
after the dot
or use non-breaking space (see below).
Example:

A fact, e.g.\ an example.

\subsubsection{Quotation Mark}

Quotation marks: ``text to be quoted'', also in `single quotes'.

\subsubsection{Dashes}

There are three types of dashes:
\begin{description}
  \item[Hyphen Dash] agent-based
  \item[Range Dash] page 123--138
  \item[em Dash] bla bla---thinking---bla bla
\end{description}

\subsubsection{Predefined special characters}

Some characters are used for special functions. If you want to write
them use the following substitutions:

\$
\&
\%
\#
\{
\}
%
\S{}
\pounds{}
\textbackslash{}
\textasciitilde{}
\textasciicircum{}
%
?`
!`

There are a lot more special characters, especially for formulas.
You will find a fairly good overview in
\href{http://www.ctan.org/tex-archive/info/lshort/english/lshort.pdf}{\texttt{lshort.pdf}}.
This is by the way a good reference for many questions concerning \LaTeX{}.

\subsubsection{Non-breaking space}

Sometimes, you do not want that two words are split at a line ending.
To prevent this use the ``tilde'' character. Example:

6~AM 6~AM 6~AM 6~AM 6~AM 6~AM 6~AM 6~AM 6~AM 6~AM 6~AM 6~AM 6~AM 6~AM
6~AM 6~AM

versus

6 AM 6 AM 6 AM 6 AM 6 AM 6 AM 6 AM 6 AM 6 AM 6 AM 6 AM 6 AM 6 AM 6 AM
6 AM 6 AM

\subsubsection{Ligatures}

Take a closer look at the words ``affluent'', ``effective'' and ``flow''.
The joining of the f and l letters
happens automatically; however on rare occasions
you may want to
suppress this, like for ``shelf{}ful''.
